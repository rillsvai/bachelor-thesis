\chapter{Постановка задачі}

\section{Описова постановка}

Нам задано довільний класичний алгоритм асиметричної криптографії та його варіацію на основі еліптичних кривих. Метою даного дослідження є розгляд та пояснення процесу адаптації класичного алгоритму до версії на основі еліптичних кривих, а також проведення ретельного і детального аналізу їх ефективності, безпеки та практичної релевантності на даний момент.

Зокрема, буде проведено порівняння обраного класичного алгоритму (Алгоритм 1) з його еквівалентною варіацією на основі еліптичних кривих (ECC) (Алгоритм 2). Аналіз охоплюватиме наступні аспекти:
\begin{itemize}
    \item \textbf{Процес адаптації:} Вивчення змін у математичній основі та алгоритмічній структурі при переході від Алгоритму 1 до Алгоритму 2.
    \item \textbf{Ефективність:} Оцінка швидкості виконання операцій шифрування та дешифрування, використання обчислювальних ресурсів та розміру ключів.
    \item \textbf{Безпека:} Аналіз стійкості алгоритмів до сучасних криптографічних атак, включаючи класичні та квантові атаки.
    \item \textbf{Практична релевантність:} Визначення придатності алгоритмів для застосування у різних умовах, таких як мобільні пристрої, вбудовані системи та великі інформаційні мережі.
\end{itemize}

Дослідження спрямоване на визначення переваг та недоліків класичних асиметричних алгоритмів у порівнянні з їх варіаціями на основі еліптичних кривих, а також надання рекомендацій щодо вибору оптимальних криптографічних методів для різних сфер застосування.
