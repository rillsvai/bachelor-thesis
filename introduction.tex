\chapter*{Вступ}
\addcontentsline{toc}{chapter}{Вступ}

У сучасному світі інформаційна безпека набуває все більшої значущості у зв’язку з постійним зростанням обсягів переданої та збереженої інформації. Захист даних від несанкціонованого доступу, зміни чи видалення є однією з ключових задач у сфері інформаційних технологій. Криптографія виступає як основний інструмент для забезпечення конфіденційності, цілісності та автентичності інформації.

\section{Історія криптографії}

Криптографія має давню історію, яка бере свій початок ще в давніх цивілізаціях. Від найпростіших методів шифрування до сучасних складних алгоритмів, криптографія розвивалась у відповідь на зростаючі потреби у безпеці інформації.

Перші спроби захисту інформації датуються часами Стародавнього Єгипту, де використовувалися прості символічні шифри для передачі секретних повідомлень. У середньовіччі арабські вчені, такі як Аль-Хорезмі, зробили значний внесок у розвиток криптографії, створюючи складніші шифри та методи їх розшифровки.

У 20 столітті криптографія отримала новий імпульс завдяки появі комп'ютерів та сучасних методів математичного аналізу. Під час Другої світової війни розвиток криптографії досягнув свого піку з розшифровкою німецьких шифрів "Енігма"  Бертраном Расселом та іншими вченими. Ці події продемонстрували важливість криптографії для військових та державних потреб.

Після численних війн криптографія почала активно застосовуватись у цивільному секторі, особливо з розвитком електронної комунікації. Виникнення Інтернету та цифрових технологій підвищило потребу у надійних методах захисту інформації, що сприяло розвитку асиметричних криптографічних алгоритмів.

\section{Сучасні криптографічні алгоритми}

Класичні алгоритми криптографії, такі як RSA та алгоритми на основі дискретного логарифму, довгий час були основою для захисту інформаційних систем. RSA (Rivest–Shamir–Adleman) є одним із найпоширеніших асиметричних алгоритмів, що використовуються для шифрування та цифрових підписів. Його безпека базується на складності задачі факторизації великих простих чисел, що робить його стійким до більшості відомих криптографічних атак.

Проте з розвитком комп’ютерних технологій та зростанням вимог до ефективності криптографічних систем виникає необхідність у пошуку більш ефективних та безпечних методів. Одним із таких напрямків є криптографія на основі еліптичних кривих (Elliptic Curve Cryptography, ECC).

\section{Криптографія на основі еліптичних кривих}

Криптографія на основі еліптичних кривих дозволяє досягати високого рівня безпеки при використанні коротших ключів порівняно з класичними методами. Це сприяє зменшенню вимог до обчислювальних ресурсів та забезпечує більш ефективну реалізацію криптографічних протоколів, що є особливо важливим для обмежених ресурсів середовищ, таких як мобільні пристрої та вбудовані системи. 

Процес адаптації класичних алгоритмів асиметричної криптографії до варіацій на основі еліптичних кривих включає заміну основної математичної задачі з факторизації великих чисел на задачу дискретного логарифму на еліптичних кривих. Це дозволяє зменшити розмір ключів та підвищити ефективність обчислювальних операцій, необхідних для шифрування, дешифрування та створення цифрових підписів. Наприклад, у випадку RSA, де для забезпечення певного рівня безпеки використовуються ключі розміром 2048 біт, ECC може забезпечити аналогічний рівень безпеки з ключами розміром лише 256 біт. Це значно зменшує обсяг збережених даних та прискорює криптографічні операції, що є критично важливим для пристроїв з обмеженими ресурсами.

\section{Мета та завдання дослідження}

Метою цієї кваліфікаційної роботи є дослідження ефективності класичних алгоритмів криптографії на основі еліптичних кривих. Зокрема, буде проведено аналіз продуктивності певного класичного алгоритму та його варіації на основі еліптичних кривих, а також оцінено їх придатність для застосування у сучасних інформаційних системах.

Завданнями дослідження є:

\begin{itemize}
    \item Ознайомлення з основними принципами та теорією криптографії потрібної для даної кваліфікаційної роботи.
    \item Аналіз існуючих класичних криптографічних алгоритмів та їх варіацій, що використовують еліптичні криві.
    \item Проведення експериментального дослідження ефективності обраних алгоритмів.
    \item Порівняння отриманих результатів та формулювання висновків щодо їх практичної застосовності.
\end{itemize}

\section{Методологія дослідження}

Для досягнення поставленої мети та вирішення завдань дослідження будуть використані наступні методи:

\begin{itemize}
    \item \textbf{Теоретичний аналіз:} Вивчення наукової літератури, математичних основ та алгоритмічних структур класичних асиметричних алгоритмів та ECC.
    \item \textbf{Порівняльний аналіз:} Оцінка ключових характеристик обох типів алгоритмів, таких як розмір ключів, обчислювальна складність, рівень безпеки.
    \item \textbf{Експериментальні методи:} Реалізація обраних алгоритмів у програмному середовищі та проведення тестів для вимірювання продуктивності та аналізу безпеки.
    \item \textbf{Статистичний аналіз:} Обробка та інтерпретація отриманих даних з метою визначення тенденцій та формулювання висновків.
\end{itemize}

\section{Очікувані результати}

Очікується, що результати дослідження дозволять визначити переваги та недоліки класичних асиметричних алгоритмів у порівнянні з їх варіаціями на основі еліптичних кривих. Зокрема, передбачається, що ECC забезпечить більш ефективне використання обчислювальних ресурсів при збереженні високого рівня безпеки, що робить його привабливим для використання у сучасних інформаційних системах. Крім того, результати можуть бути використані для оптимізації криптографічних систем, що застосовуються у різних сферах, включаючи мобільні комунікації, Інтернет речей (IoT), фінансові транзакції та захист конфіденційних даних.
