\documentclass[12pt]{report}
\usepackage[utf8]{inputenc}
\usepackage[ukrainian]{babel}
\usepackage[T1]{fontenc} 
\usepackage{graphicx}
\usepackage{amsmath, amsfonts}
\usepackage{fancyhdr} 
\usepackage{cmap}
\usepackage{tikz}

\usepackage[a4paper, margin=2.5cm]{geometry}

\fancypagestyle{plain}{
  \fancyhf{}
  \fancyhead[L]{\thepage} 
  \fancyhead[R]{Квас Кирил ПМп-43} 
  \renewcommand{\headrulewidth}{1pt}
  \renewcommand{\footrulewidth}{0pt}
  \setlength{\headheight}{14.49998pt}
}

\setcounter{tocdepth}{2}

\begin{document}

\begin{titlepage}
    \begin{center}
        \large 
        \textbf{Міністерство освіти і науки України}\\
        \textbf{Львівський національний університет імені Івана Франка}\\
        \vspace{0.5cm}
        \textbf{Факультет прикладної математики та інформатики}\\
        \textbf{Кафедра прикладної математики}\\
        
        \vfill
        
        \Large 
        \textbf{Бакалаврська робота}\\
        \vspace{0.3cm}
        \normalsize 
        \textbf{на тему:}\\
        \vspace{0.3cm}
        \Large 
        \textbf{Ефективність класичних алгоритмів криптографії \\ на основі еліптичних кривих}\\
        
        \vfill
        
        \begin{flushright}
            \normalsize 
            Виконав студент групи ПМП-43:\\
            Квас Кирил Олегович\\
            спеціальність 113 — Прикладна математика\\
            \vspace{1cm}
            Науковий керівник:\\
            доцент, кандидат фізико-математичних наук\\
            Стягар Андрій Орестович \\
            \vspace{1cm}
            Рецензент:\\
            \vspace{0.5cm}
            \underline{\hspace{7cm}}
        \end{flushright}
        
        \vfill
        
        \normalsize 
        Львів — 2024
    \end{center}
\end{titlepage}


\tableofcontents

\chapter*{Вступ}
\addcontentsline{toc}{chapter}{Вступ}

У сучасному світі інформаційна безпека набуває все більшої значущості у зв’язку з постійним зростанням обсягів переданої та збереженої інформації. Захист даних від несанкціонованого доступу, зміни чи видалення є однією з ключових задач у сфері інформаційних технологій. Криптографія виступає як основний інструмент для забезпечення конфіденційності, цілісності та автентичності інформації.

\section{Історія криптографії}

Криптографія має давню історію, яка бере свій початок ще в давніх цивілізаціях. Від найпростіших методів шифрування до сучасних складних алгоритмів, криптографія розвивалась у відповідь на зростаючі потреби у безпеці інформації.

Перші спроби захисту інформації датуються часами Стародавнього Єгипту, де використовувалися прості символічні шифри для передачі секретних повідомлень. У середньовіччі арабські вчені, такі як Аль-Хорезмі, зробили значний внесок у розвиток криптографії, створюючи складніші шифри та методи їх розшифровки.

У 20 столітті криптографія отримала новий імпульс завдяки появі комп'ютерів та сучасних методів математичного аналізу. Під час Другої світової війни розвиток криптографії досягнув свого піку з розшифровкою німецьких шифрів "Енігма"  Бертраном Расселом та іншими вченими. Ці події продемонстрували важливість криптографії для військових та державних потреб.

Після численних війн криптографія почала активно застосовуватись у цивільному секторі, особливо з розвитком електронної комунікації. Виникнення Інтернету та цифрових технологій підвищило потребу у надійних методах захисту інформації, що сприяло розвитку асиметричних криптографічних алгоритмів.

\section{Сучасні криптографічні алгоритми}

Класичні алгоритми криптографії, такі як RSA та алгоритми на основі дискретного логарифму, довгий час були основою для захисту інформаційних систем. RSA (Rivest–Shamir–Adleman) є одним із найпоширеніших асиметричних алгоритмів, що використовуються для шифрування та цифрових підписів. Його безпека базується на складності задачі факторизації великих простих чисел, що робить його стійким до більшості відомих криптографічних атак.

Проте з розвитком комп’ютерних технологій та зростанням вимог до ефективності криптографічних систем виникає необхідність у пошуку більш ефективних та безпечних методів. Одним із таких напрямків є криптографія на основі еліптичних кривих (Elliptic Curve Cryptography, ECC).

\section{Криптографія на основі еліптичних кривих}

Криптографія на основі еліптичних кривих дозволяє досягати високого рівня безпеки при використанні коротших ключів порівняно з класичними методами. Це сприяє зменшенню вимог до обчислювальних ресурсів та забезпечує більш ефективну реалізацію криптографічних протоколів, що є особливо важливим для обмежених ресурсів середовищ, таких як мобільні пристрої та вбудовані системи. 

Процес адаптації класичних алгоритмів асиметричної криптографії до варіацій на основі еліптичних кривих включає заміну основної математичної задачі з факторизації великих чисел на задачу дискретного логарифму на еліптичних кривих. Це дозволяє зменшити розмір ключів та підвищити ефективність обчислювальних операцій, необхідних для шифрування, дешифрування та створення цифрових підписів. Наприклад, у випадку RSA, де для забезпечення певного рівня безпеки використовуються ключі розміром 2048 біт, ECC може забезпечити аналогічний рівень безпеки з ключами розміром лише 256 біт. Це значно зменшує обсяг збережених даних та прискорює криптографічні операції, що є критично важливим для пристроїв з обмеженими ресурсами.

\section{Мета та завдання дослідження}

Метою цієї кваліфікаційної роботи є дослідження ефективності класичних алгоритмів криптографії на основі еліптичних кривих. Зокрема, буде проведено аналіз продуктивності певного класичного алгоритму та його варіації на основі еліптичних кривих, а також оцінено їх придатність для застосування у сучасних інформаційних системах.

Завданнями дослідження є:

\begin{itemize}
    \item Ознайомлення з основними принципами та теорією криптографії потрібної для даної кваліфікаційної роботи.
    \item Аналіз існуючих класичних криптографічних алгоритмів та їх варіацій, що використовують еліптичні криві.
    \item Проведення експериментального дослідження ефективності обраних алгоритмів.
    \item Порівняння отриманих результатів та формулювання висновків щодо їх практичної застосовності.
\end{itemize}

\section{Методологія дослідження}

Для досягнення поставленої мети та вирішення завдань дослідження будуть використані наступні методи:

\begin{itemize}
    \item \textbf{Теоретичний аналіз:} Вивчення наукової літератури, математичних основ та алгоритмічних структур класичних асиметричних алгоритмів та ECC.
    \item \textbf{Порівняльний аналіз:} Оцінка ключових характеристик обох типів алгоритмів, таких як розмір ключів, обчислювальна складність, рівень безпеки.
    \item \textbf{Експериментальні методи:} Реалізація обраних алгоритмів у програмному середовищі та проведення тестів для вимірювання продуктивності та аналізу безпеки.
    \item \textbf{Статистичний аналіз:} Обробка та інтерпретація отриманих даних з метою визначення тенденцій та формулювання висновків.
\end{itemize}

\section{Очікувані результати}

Очікується, що результати дослідження дозволять визначити переваги та недоліки класичних асиметричних алгоритмів у порівнянні з їх варіаціями на основі еліптичних кривих. Зокрема, передбачається, що ECC забезпечить більш ефективне використання обчислювальних ресурсів при збереженні високого рівня безпеки, що робить його привабливим для використання у сучасних інформаційних системах. Крім того, результати можуть бути використані для оптимізації криптографічних систем, що застосовуються у різних сферах, включаючи мобільні комунікації, Інтернет речей (IoT), фінансові транзакції та захист конфіденційних даних.


\chapter{Постановка задачі}

\section{Описова постановка}

Нам задано довільний класичний алгоритм асиметричної криптографії та його варіацію на основі еліптичних кривих. Метою даного дослідження є розгляд та пояснення процесу адаптації класичного алгоритму до версії на основі еліптичних кривих, а також проведення ретельного і детального аналізу їх ефективності, безпеки та практичної релевантності на даний момент.

Зокрема, буде проведено порівняння обраного класичного алгоритму (Алгоритм 1) з його еквівалентною варіацією на основі еліптичних кривих (ECC) (Алгоритм 2). Аналіз охоплюватиме наступні аспекти:
\begin{itemize}
    \item \textbf{Процес адаптації:} Вивчення змін у математичній основі та алгоритмічній структурі при переході від Алгоритму 1 до Алгоритму 2.
    \item \textbf{Ефективність:} Оцінка швидкості виконання операцій шифрування та дешифрування, використання обчислювальних ресурсів та розміру ключів.
    \item \textbf{Безпека:} Аналіз стійкості алгоритмів до сучасних криптографічних атак, включаючи класичні та квантові атаки.
    \item \textbf{Практична релевантність:} Визначення придатності алгоритмів для застосування у різних умовах, таких як мобільні пристрої, вбудовані системи та великі інформаційні мережі.
\end{itemize}

Дослідження спрямоване на визначення переваг та недоліків класичних асиметричних алгоритмів у порівнянні з їх варіаціями на основі еліптичних кривих, а також надання рекомендацій щодо вибору оптимальних криптографічних методів для різних сфер застосування.

\section{Математична модель}

Рисунок \ref{fig:basic_protocol_classical} і \ref{fig:basic_protocol_ecc} демонструє процес обміну зашифрованими повідомленнями між двома сторонами: відправником (Алісою) та отримувачем (Бобом) для клачисного і аналогу на основі еліптичних кривих відповідно


\begin{figure}[htbp]
\begin{center}
\begin{tikzpicture}

% Alice and Bob nodes
\node at (-2, 0) {\textbf{Аліса}};
\node at (8, 0) {\textbf{Боб}};

% Encryption and decryption
\node at (-0.9, -1) {$y = e_{k_{\text{pub}}^{(1)}}(x)$};
\node at (6.9, -1) {$x = d_{k_{\text{pr}}^{(1)}}(y)$};

% Key pair generation
\node at (7.5, 1) {$(k_{\text{pub}}^{(1)}, k_{\text{pr}}^{(1)}) = k^{(1)}$};

% Arrows and labels
\draw[<-] (0.5, 1) -- (5.5, 1) node[midway, above] {$k_{\text{pub}}^{(1)}$};
\draw[->] (0.5, -1) -- (5.5, -1) node[midway, below] {$y^{2}$};

\end{tikzpicture}
\caption{Базовий протокол асиметричного шифрування на основі класичного алгоритму.}
\label{fig:basic_protocol_classical}
\end{center}
\end{figure}

\begin{figure}[htbp]
\begin{center}
\begin{tikzpicture}

% Alice and Bob nodes
\node at (-2, 0) {\textbf{Аліса}};
\node at (8, 0) {\textbf{Боб}};

% Encryption and decryption
\node at (-0.9, -1) {$y = e_{k_{\text{pub}}^{(2)}}(x)$};
\node at (6.9, -1) {$x = d_{k_{\text{pr}}^{(2)}}(y)$};

% Key pair generation
\node at (7.5, 1) {$(k_{\text{pub}}^{(2)}, k_{\text{pr}}^{(2)}) = k^{(2)}$};

% Arrows and labels
\draw[<-] (0.5, 1) -- (5.5, 1) node[midway, above] {$k_{\text{pub}}^{(2)}$};
\draw[->] (0.5, -1) -- (5.5, -1) node[midway, below] {$y^{2}$};

\end{tikzpicture}
\caption{Базовий протокол асиметричного шифрування на основі ECC.}
\label{fig:basic_protocol_ecc}
\end{center}
\end{figure}

\begin{itemize}
    \item $k_{\text{pub}}^{(1)}$, $k_{\text{pub}}^{(2)}$: публічний ключ для класичного алгоритму та ECC відповідно.
    \item $k_{\text{pr}}^{(1)}$, $k_{\text{pr}}^{(2)}$: приватний ключ для класичного алгоритму та ECC відповідно.
    \item $k^{(1)}$, $k^{(2)}$: пара ключів (публічний і приватний) для класичного алгоритму та ECC відповідно.
    \item $e_{k_{\text{pub}}^{(1)}}$, $e_{k_{\text{pub}}^{(2)}}$: функція шифрування з використанням публічного ключа.
    \item $d_{k_{\text{pr}}^{(1)}}$, $d_{k_{\text{pr}}^{(2)}}$: функція дешифрування з використанням приватного ключа.
    \item $x$: вихідне повідомлення, яке потрібно зашифрувати.
    \item $y^{(1)}, y^{(2)}$: зашифроване повідомлення з використанням класичного алгоритму та ECC відповідно.
\end{itemize}

\chapter{Проблема дискретного логарифму}

\section{Огляд головних криптографічних схем}

Сучасна криптографія спирається на три основні сімейства алгоритмів відкритого ключа, які є практично значущими та базуються на різних математичних проблемах. Кожна з цих схем забезпечує такі основні криптографічні функції, як встановлення ключів, цифрові підписи для нерепудійованості та шифрування даних. У цьому розділі наведено короткий огляд цих трьох сімейств, спираючись на матеріали книги.

\subsection{Схеми на основі факторизації цілих чисел}

Схеми цієї групи ґрунтуються на складності задачі факторизації великих цілих чисел. Найвідомішим представником цього сімейства є алгоритм RSA, який був запропонований у 1977 році. RSA широко використовується для шифрування, цифрових підписів і встановлення ключів. Ефективність RSA залежить від обчислювальної складності розкладу великих чисел на прості множники. При виборі належних параметрів, таких як довжина ключа, алгоритм забезпечує високий рівень безпеки.

\subsection{Схеми на основі дискретного логарифму}

Алгоритми цього сімейства базуються на задачі дискретного логарифму в скінченних полях. До найвідоміших представників належать:
\begin{itemize}
    \item Протокол обміну ключами Діффі–Геллмана (Diffie–Hellman key exchange),
    \item Шифрування Ель-Гамаля (ElGamal encryption),
    \item Алгоритм цифрового підпису (Digital Signature Algorithm, DSA).
\end{itemize}

Ці алгоритми були запропоновані в середині 1970-х років і залишаються надійними за умови правильного вибору параметрів. В основі їх безпеки лежить складність обчислення дискретного логарифму — задачі, для якої не існує відомих ефективних алгоритмів розв’язання.

\subsection{Схеми на основі еліптичних кривих}

Ця група є узагальненням схем на основі дискретного логарифму. Алгоритми на основі еліптичних кривих (Elliptic Curve Cryptography, ECC) були запропоновані в середині 1980-х років і мають перевагу у зменшенні розмірів ключів без втрати рівня безпеки. Найвідоміші приклади включають:
\begin{itemize}
    \item Обмін ключами за допомогою еліптичних кривих (Elliptic Curve Diffie–Hellman, ECDH),
    \item Алгоритм цифрового підпису на основі еліптичних кривих (Elliptic Curve Digital Signature Algorithm, ECDSA).
\end{itemize}

Схеми ECC використовують коротші ключі порівняно з класичними алгоритмами, такими як RSA, забезпечуючи такий самий рівень криптографічної безпеки. Це робить ECC привабливими для середовищ з обмеженими ресурсами, наприклад, мобільних пристроїв або вбудованих систем.

\subsection{Інші схеми відкритого ключа}

Окрім трьох основних сімейств, існують також інші схеми, такі як:
\begin{itemize}
    \item Мультиваріативні квадратичні схеми (Multivariate Quadratic, MQ),
    \item Схеми на основі ґраток (Lattice-based schemes),
    \item Криптосистеми McEliece.
\end{itemize}

Проте ці схеми часто мають недостатню криптографічну зрілість або погані характеристики реалізації, наприклад, надмірно великі ключі. Інші схеми, наприклад, криптосистеми на основі гіпереліптичних кривих, є як ефективними, так і безпечними, але поки що не набули широкого розповсюдження. Для більшості застосувань рекомендовано використовувати схеми з трьох основних сімейств.

Таким чином, головними сімействами алгоритмів відкритого ключа, які забезпечують надійний захист і практичність, є схеми на основі факторизації цілих чисел, дискретного логарифму та еліптичних кривих.

\end{document}


